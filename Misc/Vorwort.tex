\section*{Vorwort}

\begin{flushright}
    Hamburg, 6. Mai 2025
\end{flushright}


\bf Liebe Mitglieder des Stammes Johannes Bugenhagen! \sf \\

Nun ist es fertig, das nagelneue StaLiBu 2! Als Grundlage für dieses Buch dient das „Pfadiralala IV“ der Region Kurhessen im VCP - ein riesen Dank geht an alle fleißigen Hände im Impressum, welche bereits viele Lieder erstellt und mir damit die Arbeit sehr erleichtert haben. \\

Ich habe noch weitere Lieder hinzugefügt und Akkorde korrigiert. Einige Lieder hatten einen leicht anderen Text, als wir ihn kennen. Bei diesen Liedern habe ich den Text an den im JBN bekannten angepasst. \\

Ein letzter Dank gilt Jakob, welcher mich in LaTeX eingearbeitet hat und mir oft weiterhelfen konnte - Die Liedbucherstellung war so sehr angenehm. Von ihm stammt beispielsweise das nach Liedtiteln und Liedanfang visuell abgegrenzte Inhaltsverzeichnis. \\

Für Verbesserungsvorschläge oder Hinweise schreibt mir gern unter vincent@jonbu.de. \\ \\

\bf Hinweise zum StaLiBu 2 \sf \\

Die Lieder sind in 4 Kategorien (Pfadfinder, Ärzte, Deutsch, Englisch) eingeteilt und jeweils alphabetisch nach Liedtitel sortiert. Genutzt werden die deutschen Notennamen (C, D, E, F, G, A, H) mit 'B' für 'Hb'. \newline
Die \_ sind zur Lesbarkeit da und geben an, dass ein Akkord länger gehalten wird. Die Lieder sind nie länger als eine Doppelseite - wenn es nötig war, habe ich dafür die Noten entfernt. Der Refrain wird, wenn der Platz reicht, immer voll ausgeschrieben. Ansonsten wird ''Refrain (wdh.)'' angegeben. Der Refrain ist dann meist auch zu Beginn der 2. Seite zu finden. \\

Ich habe die Lieder an das Singen mit einer Gitarre angepasst. Teilweise habe ich die Akkorde für Soli/Zwischenspiele gekürzt oder entfernt. Bei einigen Liedern (z.B. solche mit Fade-Out) habe ich das Ende so verändert, dass man diese gut mit Gitarre abschließen kann.
Mit den Capo-Angaben können die Lieder in Original-Tonart gespielt werden. Capo -1 bedeutet, dass die Saiten der Gitarre um einen Halbton runter gestimmt werden müssen. \\

Viel Spaß mit dem StaLiBu 2 und Gut Pfad wünscht euch\\
Vincent / Headset