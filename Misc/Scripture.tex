\section*{Zu den Liedern}

\usefont{OT1}{pzc}{mb}{it}

Drei rote Pfiffe\\[-0.5em]
\noindent\rule{\textwidth}{0.2pt}
Helena Kuchar (1906-1985) war im abgelegenen Lepena-Tal unter dem Namen ''Jelka'' (slowenisch für ''Tanne'') als Partisanin an der Befreiung Österreichs vom Nationalsozialismus beteiligt.
Während die Männer sich im Wald und dem Gebirge versteckt hielten, oragnisierte Jelka Nahrumgsmittel, Kleidung und Medikamente.
Zudem baute sie ein Nachrichtennetzwerk zur geheimen Kommunikation auf. Mit den ''drei roten Pfiffen'' gab Jelka sich als Verbündete zu erkennen. \\[-0.5em]
\noindent\rule{\textwidth}{0.2pt} \\

Edelweißpiraten\\[-0.5em]
\noindent\rule{\textwidth}{0.2pt}
Die Edelweißpiraten waren eine Gruppe Jugendlicher zur Zeit des Dritten Reichs, die sich der Hitlerjugend nicht anschließen wollten.
Als sie weiterhin auf Fahrten gingen, wurden sie von der Gestapo verfolgt und wurden oppositionell politisch.
Bis zuletzt wurden Edelweißpiraten, die an einem Abzeichen erkennbar waren, verfolgt und erschossen.
Bis heute sind sie nicht als Widerstandsgruppe anerkannt. Dieses Lied reflektiert besonders die Bedeutung der Ehrenfelder Gruppe. \\[-0.5em]
\noindent\rule{\textwidth}{0.2pt} \\

Hester Jonas\\[-0.5em]
\noindent\rule{\textwidth}{0.2pt}
Die sog. Hexe von Neuss (Ort bei Düsseldorf), geboren um 1570, wurde wegen Gerüchten verhaftet und verhört.
Man warf der Hebamme und Kräuterheilkundlerin Schadenzauber, den Abfall von Gott und einen Pakt mit dem Teufel vor, was sie nach brutaler Folter auch bestätigte.
Nach zwischenzeitlicher Flucht wurde sie am 24. Dezember 1635 enthauptet. \\[-0.5em]
\noindent\rule{\textwidth}{0.2pt} \\

Jalava\\[-0.5em]
\noindent\rule{\textwidth}{0.2pt}
Lenin (Wladimir Iljitsch Uljanow) floh im Juli 1917 nach einem missglückten Aufstand (sog. Julitage) über die russische Grenze nach Finnland.
Von dort sandte er Anweisungen an seine Verbündeten in Russland und bereitete so die Oktoberrevolution vor.
Erst kurz vor Ausbruch der Revolution am 25. Oktober 1917 kehrte Lenin heimlich nach Petersburg zurück.
Das Lied erzählt, wie Lenin sich, als Heizer verkleidet, mit der Lokomotive des finnischen Lokführers (und Kommunisten) Hugo Erikowitsch Jalava über die Grenze schmuggeln lässt.
Die Heizer-Schmuggelei fand dabei eigentlich bei der Flucht aus Russland, nicht bei der Rückkehr nach Petersburg, statt; bei dieser verkleidete sich Lenin als finnischer lutheranischer Pastor, mit Perücke und abrasiertem Bart. \\[-0.5em]
\noindent\rule{\textwidth}{0.2pt}