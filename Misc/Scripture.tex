\section*{Erläuterungen zu den Liedern}

\usefont{OT1}{pzc}{mb}{it}
Drei rote Pfiffe\\[-0.5em]
\noindent\rule{\textwidth}{0.5pt}
Helena Kuchar (1906-1985) war im abgelegenen Lepena-Tal unter dem Namen ''Jelka'' (slowenisch für ''Tanne'') als Partisanin an der Befreiung Österreichs vom Nationalsozialismus beteiligt.
Während die Männer sich im Wald und dem Gebirge versteckt hielten, oragnisierte Jelka Nahrumgsmittel, Kleidung und Medikamente.
Zudem baute sie ein Nachrichtennetzwerk zur geheimen Kommunikation auf. Mit den ''drei roten Pfiffen'' gab Jelka sich als Verbündete zu erkennen.

