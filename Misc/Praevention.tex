\definecolor{beige}{HTML}{F7E2B2}
\pagecolor{beige}

\section*{Präventionsgrundsätze}

Die folgenden Grundsätze können Kinder und Jugendliche gegen sexualisierte Gewalt stärken. Sie zu vermitteln ist Grundlage der Präventionsarbeit.
Sie können sehr gut in Gruppenstunden mit verschiedenen Altersgruppen eingesetzt werden. Sie sind jedoch keine Garantie dafür, dass ein Kind oder ein*e Jugendliche*r keine sexualisierte Gewalt erlebt. Sie ersetzen
auch nicht die Verantwortung Erwachsener, die sexualisierte Gewalt befürchten oder beobachten. \\ \\
+ Dein Körper gehört dir! \\[5pt]
+ Du bist wichtig und du hast das Recht zu bestimmen, wie, wann, wo und von wem
\noindent\hspace*{2mm} du angefasst werden möchtest. \\[5pt]
+ Deine Gefühle sind wichtig! \\[5pt]
+ Du kannst deinen Gefühlen vertrauen. Es gibt angenehme Gefühle, da fühlst du
\noindent\hspace*{2mm} dich gut und wohl. Unangenehme Gefühle sagen dir, dass etwas nicht stimmt, du
\noindent\hspace*{2mm} fühlst dich komisch. Sprich über deine Gefühle, auch wenn es schwierige Gefühle
\noindent\hspace*{2mm} sind. \\[5pt]
+ Es gibt angenehme und unangenehme Berührungen! \\[5pt]
+ Es gibt Berührungen, die sich gut anfühlen und richtig glücklich machen. Aber es
\noindent\hspace*{2mm} gibt auch solche, die komisch sind, Angst auslösen oder sogar weh tun. Niemand
\noindent\hspace*{2mm} hat das Recht, dich zu schlagen oder dich so zu berühren, wie und wo du es nicht
\noindent\hspace*{2mm} willst. Manche Leute möchten so berührt werden, wie du es nicht willst. Niemand
\noindent\hspace*{2mm} darf dich zu Berührungen überreden oder zwingen. \\[5pt]
+ Du hast das Recht, Nein zu sagen! \\[5pt]
+ Es gibt Situationen, in denen du nicht gehorchen musst. \\[5pt]
+ Es gibt gute und blöde Geheimnisse! \\[5pt]
+ Gute Geheimnisse machen Freude und sind spannend. Blöde Geheimnisse sind
\noindent\hspace*{2mm} unheimlich und schwer zu ertragen. Solche darfst du weitererzählen, auch wenn
\noindent\hspace*{2mm} du versprochen hast, es niemandem zu sagen. \\[5pt]
+ Sprich darüber, hole Hilfe! \\[5pt]
+ Wenn dich etwas bedrückt oder du unangenehme Erlebnisse hast, rede darüber
\noindent\hspace*{2mm} mit einer Person, der du vertraust. Höre nicht auf zu erzählen, bis dir geholfen wird. \\[5pt]
+ Du bist nicht schuld! \\[5pt]
+ Wenn Erwachsene deine Grenze überschreiten – egal, ob du Nein sagst oder nicht
\noindent\hspace*{2mm} – sind immer die Erwachsenen verantwortlich für das, was passiert.