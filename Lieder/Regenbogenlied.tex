\beginsong{Regenbogenlied}[
    index={Von überall sind wir gekommen},
    wuw={P. Kümmel, J. Geißler, C. Ette, P. Steinacher, BdP},
    jahr={1976},
]

\beginverse
\endverse
\includegraphics[draft=false, width=1\textwidth]{Noten/Regenbogenlied.pdf}

\beginverse\memorize
\[C] Von Süden, Osten, \[F]West und Norden,
\[G] sind wir vereint zum \[C]großen Spiel,
\[Am] denn weit ist unser \[Dm]Kreis geworden
\[G] und nur in ihm liegt \[C]unser Ziel.
\endverse

\beginchorus
\[F] Über uns ein \[G]Regenbogen,
\[C] zeigt uns den Weg in \[Am]seinem Licht.
\[F] Die Wolken sind schon \[G]fortgezogen,
\[C] verwehren \[Dm]uns die \[G7]Sonne \[C]nicht.
\endchorus

\beginverse
^ Und abends in der ^Lagerrunde,
^ erzählen wir von ^dir und mir,
^ scheint auch kein Licht in ^dieser Stunde,
^ am nächsten Morgen ^wissen wir:
\endverse

\beginchorus
\lrep \[F] Sind wir einmal \[G]fortgezogen,
\[C] dorthin, wo es \[Am]uns gefällt,
\[F] bringt auch unser \[G]Regenbogen
\[C] neue \[Dm]Farben \[G7]in die \[C]Welt.\rrep
\endchorus

\endsong

\beginscripture{}
Bundeslagerlied des BdP für das Bundeslager 1977 in Kirchberg unter dem Motto~ ''Regenbogen''.
\endscripture
