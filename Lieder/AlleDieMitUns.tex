\beginsong{Alle, die mit uns auf Kaperfahrt fahren}[
    txt={Gottfried Wolters}, 
    txtjahr={1951},
    mel={aus Flandern},
    bo={12},
    pfii={108},
    kssiv={64},
    siru={8},
]

\beginverse
\endverse
\includegraphics[draft=false, width=1\textwidth]{Noten/Lied058.pdf}	

\beginverse
\[Em]Alle, die \[H7]Tod und \[Em]Teufel nicht fürchten,
müssen \[H7]Männer mit \[Em]Bärten sein.
\endverse

\beginchorus
\[G]Jan und Hein und Klaas und Pitt,
\[Em]die haben \[H7]Bärte, die haben \[Em]Bärte,
\[G]Jan und Hein und Klaas und Pitt,
\[Em]die haben \[H7]Bärte, die \[Em]fah\[H7]ren \[Em]mit.
\endchorus

\beginverse 
^Alle, die ^Weiber und ^Branntwein lieben,
müssen ^Männer mit ^Bärten sein.
\endverse

\printchorus

\beginverse
^Alle, die ^mit uns das ^Walross killen,
müssen ^Männer mit ^Bärten sein.
\endverse

\beginchorus
\[G]Jan und Hein und Klaas und Pitt,
\[Em]die haben \[H7]Bärte, die haben \[Em]Bärte,
\[G]Jan und Hein und Klaas und Pitt,
\[Em]die haben \[H7]Bärte, die \[Em]fah\[H7]ren \[Em]mit.
\endchorus

\beginverse
^Alle, die ^öligen ^Zwieback lieben,
müssen ^Männer mit ^Bärten sein.
\endverse

\printchorus

\beginverse
^Alle, die ^endlich zur ^Höll' mitfahren,
müssen ^Männer mit ^Bärten sein.
\endverse

\printchorus

\endsong

\beginscripture{}
Ein deutsches Seemannslied, das auf dem flämischen Volkslied~ ''Al die willen te kap'ren varen'' basiert. Die ursprüngliche flämische Version wurde in der ersten Hälfte des 19. Jahrhunderts von Edmond de Coussemaker aufgezeichnet und erschien 1856 in der Sammlung~ ''Chants populaires des Flamands de France''.
\endscripture
