\beginsong{Bündische Vaganten}[
    wuw={Trenk (Alo Hamm)},
    jahr={1952},
    bo={178}, 
    pfii={27}, 
    pfiii={58}, 
    siru={96}, 
    tonspur={292}, 
    index={Hej, wie vorn der Fetzen fliegt},
]

\beginverse
\endverse
\includegraphics[draft=false, width=1\textwidth]{Noten/Lied011.pdf}

\beginverse
\[Em]Treiben wir dem \[H7]Süden zu, lässt \[Em]uns der Norden \[H7]keine Ruh',
\[Am]über\[Em]all zu \[H7]Haus' sind \[Em]wir.
Mal \[Em]rüber nach A\[H7]merika, mal \[Em]runter bis nach \[H7]Afrika,
\[Am]hoja, \[Em]hoja, \[H7]das sind \[Em]wir!
\endverse

\beginchorus 
\[C]Bündische Va\[G]ganten
\[D7]tippeln in die Welt, \[G]tippeln in die Welt.
\[C]Bündische Va\[G]ganten
\[D7]tippeln in die Welt, hei-o A\[G]yen!
\endchorus

\beginverse
^Hast du noch ein ^jung' Gesicht, so ^zage nicht und ^fack'le nicht, 
^frage ^niemals ^nach dem ^'Wie?'
Wer ^nur am Rand der ^Straße klebt, für ^seinen dummen ^Bauch nur lebt,
^misst der ^Ferne ^Zauber ^nie.
\endverse

\beginchorus
\[C]Bündische Va\[G]ganten
\[D7]tippeln in die Welt, \[G]tippeln in die Welt.
\[C]Bündische Va\[G]ganten
\[D7]tippeln in die Welt, hei-o A\[G]yen!
\endchorus

\endsong

\beginscripture{}
Vagant = Fahrendes Volk/Herumziehender; Das Lied behandelt die deutsche Jugendbewegung zwischen 1919 und 1933, die aus dem Wandervogel entstand und die Hinwendung der städtischen bürgerlichen Jugend zum Naturleben meint.
\endscripture
