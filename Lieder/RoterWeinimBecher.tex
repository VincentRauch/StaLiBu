\beginsong{Roter Wein im Becher}[
    wuw={Helmut König, Musik nach einem französischen Volkslied}, 
    bo={268}, 
    pfii={15}, 
    pfiii={9}, 
    gruen={76}, 
    kssiv={25}, 
    siru={195},
    tonspur={438}, 
]

\beginverse
\endverse
\includegraphics[draft=false, width=1\textwidth]{Noten/Lied079.pdf}	

\beginverse
\[Em]Morgens bricht die Runde zu \[D7]neuen Fahrten \[G]auf.
Es \[D7]klingt in aller \[G]Mun\[Am]de ein \[Em]frohes \[H7]Liedchen \[Em]auf. 
\endverse

\beginchorus
\lrep Radi, radi, ra-di \[C]rala\[G]la, \[D7]radi, radi, \[Em]ra-di \[H7]ralalala\[Em]la. \rrep
\endchorus

\beginverse
^Steine, Staub und Dornen sind ^schwerlich Tippe^lei.
Wir ^müssen uns an^spor^nen, die ^Qual ist ^bald vor^bei.
\endverse

\printchorus

\beginverse
^Treffen wir uns wieder, der ^Zufall nennt den ^Ort,
so ^schallen uns're ^Lie^der in ^weite ^Ferne ^fort.
\endverse

\printchorus

\endsong

\beginscripture{}
Tippelei = Walz, Gesellenwanderung
\endscripture
