\beginsong{Horizont-Lied}[
    index={Ich bin viel rumgekommen},
]

\beginverse
\endverse
\includegraphics[draft=false, width=1\textwidth]{Noten/Horizont.pdf}

\beginverse
\[G]Wo ist dein zu\[C]hause, von \[D]wie weit kommst du \[G]her?
Und wie es sich so \[C]lebt bei dir, das \[Am]interessiert mich \[D]sehr.
\[G]Was ist bei dir \[C]anders? \[D]Was dir so ge\[G]fällt.
Das \[G]will ich wirklich \[F]wissen - komm \[C]zeig mir deine \[D]Welt.
Die \[C]will ich gerne \[Em]seh'n und \[G]will dich ver\[D]stehn.
\endverse

\beginchorus
\[G] Komm' \[D]los, \[C]in ein neues Land.\[G]
\[]\[]\[] Nicht al\[D]lein, wir \[C]gehen Hand in \[G]Hand.
Er\[Em]kunden wir den \[Am]Horizont, geh'n \[C]weiter als bis\[D]her.
\[G]Alles zu ent\[C]decken und \[D]morgen noch viel \[G]mehr.
\endchorus

\beginverse
Den ^Horizont als ^Zeichen, uner^reichbar und doch ^da.
Wenn ^Himmel sich und ^Erde treffen,^ist Gott für mich ^nah.
Wie ^diese dünne ^Linie, die die ^ganze Welt um^fasst,
so ^ist Gott immer ^da für mich, auch ^wenn's mir grad nicht ^passt.
Ein ^Zelt ist Gottes ^Hand - ^über mir ge\[D]spannt.
\endverse

\beginchorus
\[G] Komm' \[D]los, \[C]in ein neues Land.\[G]
\[]\[]\[] Nicht al\[D]lein, wir \[C]gehen Hand in \[G]Hand.
Er\[Em]kunden wir den \[Am]Horizont, geh'n \[C]weiter als bis\[D]her.
\[G]Alles zu ent\[C]decken und \[D]morgen noch viel \[G]mehr.
\endchorus

\endsong

\beginscripture{}
Bundeslagerlied des BdP für das Bundeslager 1993 in Friedeburg unter dem Motto~ ''Über den Horizont''.
\endscripture