\beginsong{Digeding Dong Dong}

\beginverse
\endverse
\includegraphics[draft=false, width=1\textwidth]{Noten/DigedingDongDong.pdf}

\beginverse\memorize
Dige\[Em]ding dong dong, elles s'en vont à con\[D]fesse
\[Em]au curé du can\[D]ton.
\endverse

\beginverse
Dige^ding dong dong, qu'avez vous fait, les ^filles
\[Em]pour demander par\[D]don?
\endverse

\beginverse
Dige^ding dong dong, j'avions couru les ^bals
\[Em]et les jolis gar\[D]çons.
\endverse

\beginverse
Dige^ding dong dong, ma fille pour péni^tence
\[Em]nous nous embrasse\[D]rons.
\endverse

\beginverse
Dige\[Em]ding dong dong, je n'embrasse point les \[D]prêtres,
\[Em]mais les jolis gar\[D]çons
\[Em]qu'ont du poil au men\[D]ton.
\endverse

\beginverse
Dige^ding dong dong, c'est sont les filles des ^forges.
\[Em]Des forges de Paim\[D]pont.
\endverse

\endsong

\beginscripture{}
Ein traditionelles bretonisches Volkslied, das seinen Ursprung in der Region Paimpont in der Bretagne, Frankreich hat. Die erste bekannte Version des Liedes wurde im März 1872 von dem bretonischen Folkloristen Adolphe Orain im Dorf Le Cannée in der Gemeinde Paimpont gesammelt. Eine zweite Version wurde etwa 1884 von einem Schuhmacher aus Paimpont beigesteuert. Eine dritte Version wurde in den 1970er Jahren von der Gruppe Tri Yann populär gemacht und erschien 1976 auf ihrem ersten Album. \\

Übersetzung: Sie sind die Töchter der Schmiede, der Schmiede von Paimpont. / Sie gehen zur Beichte, zum örtlichen Pfarrer. /~ ''Was habt ihr Mädchen getan, um um Vergebung zu bitten?'' /~ ''Ich bin auf den Tanzbällen den hübschen Jungen nachgelaufen.'' /~ ''meine Tochter, zur Buße werden wir uns küssen.'' /~ ''Ich küsse nicht die Priester, sondern die schönen Jungen, die Haare am Kinn haben.''
\endscripture