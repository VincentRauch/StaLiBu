\beginsong{Über den Wolken}[
    wuw={Reinhard Mey}, 
    alb={Wie vor Jahr und Tag}, 
    jahr={1974}, 
    pfii={174}, 
    pfiii={90}, 
    kssiv={120}, 
    index={Wind Nord-Ost, Startbahn},
]

\beginverse
\[G] Wind Nord-Ost, Startbahn null-\[Am]drei,\[D] bis hier hör' ich die Mo\[G]toren,
wie ein Pfeil zieht sie \[Am]vorbei\[D] und es dröhnt in meinen \[G]Ohren.
Und der nasse Asphalt \[Am]bebt,\[D] wie ein Schleier staubt der \[G]Regen,
bis sie abhebt und sie \[Am]schwebt\[D] der Sonne ent\[G]gegen.
\endverse

\beginchorus
Über den \[C]Wolken\[D] muss die Freiheit wohl \[G]grenzenlos sein!\[Em]
Alle Ängste, alle \[Am]Sorgen sagt man,\[D] blieben darunter ver\[G]borgen und dann\[C]
würde was uns groß und \[G]wichtig erscheint\[D] plötzlich nichtig und \[G]klein.
\endchorus

\beginverse
^ Ich seh' ihr noch lange ^nach,^ seh' sie die Wolken er^klimmen,
bis die Lichter ^nach^ und nach ganz im Regengrau ver^schwimmen.
Meine Augen haben ^schon^ jenen winz'gen Punkt ver^loren,
Nur von fern klingt mono^ton ^ das Summen der ^Motoren.
\endverse

\printchorus

\beginverse
^ Dann ist alles still, ich ^geh'^, Regen durchdringt meine ^Jacke,
irgendjemand kocht Ka^ffee ^ in der Luftaufsichtsba^racke.
In den Pfützen schwimmt Ben^zin, ^ schillernd wie ein Regen^bogen.
Wolken spiegeln sich da^rin, ^ Ich wär' gern mitge^flogen.
\endverse

\printchorus

\endsong
