\beginsong{Der Piet}[
    wuw={Mac (Erik Martin)},
    jahr={1981}, 
    bo={364}, 
    pfii={69}, 
    pfiii={25}, 
    gruen={196}, 
    kssiv={46}, 
    siru={252}, 
    tonspur={570},
    index={Was kann ich denn dafür?},
]

\beginverse
\endverse
\includegraphics[draft=false, width=1\textwidth]{Noten/Lied019.pdf}

\beginverse
Sie \[Am]nahmen mir die Schuh' und \[C]auch den Rock dazu.
Sie \[G]banden mir die Händ' und mein \[Am]Haus, es hat gebrennt.
Ich \[Am]sah den Galgen steh'n. Sie \[C]zwangen mich zu geh'n.
Sie \[G]wollten meinen Tod, keiner \[Am]half mir in der Not.
\endverse

\beginchorus
\lrep \[Am]Wenn der Nebel auf das \[D]Moor sich senkt, der \[F]Piet am \[G]Galgen \[Am]hängt.\rrep
\endchorus

\newpage

\beginverse
Was ^kratzt da am Genick? Ich ^spür' den rauhen Strick.
Ein ^Mönch der betet dort und spricht ^für mich fromme Wort,
die ^Wort, die ich nicht kenn', wer ^lehrte sie mich denn? 
Fünf ^Raben fliegen her, doch ich ^sehe sie nicht mehr.
\endverse

\beginchorus
\lrep \[Am]Wenn der Nebel auf das \[D]Moor sich senkt, der \[F]Piet am \[G]Galgen \[Am]hängt.\rrep
\endchorus

\endsong
