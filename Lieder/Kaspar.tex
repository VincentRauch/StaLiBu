\beginsong{Kaspar}[
    wuw={Reinhard Mey}, 
    jahr={1968}, 
    pfii={94}, 
    pfiii={45}, 
    gruen={18}, 
    kssiv={68}, 
    siru={212}, 
    tonspur={462}, 
    index={Sie sagten, er käme von Nürnberg her},
]

\beginverse\memorize
Sie \[Am]sagten, er käme von \[D]Nürnberg her und er \[Am]spräche kein Wort.
Auf dem Marktplatz standen sie \[D]um ihn her und be\[Am]gafften ihn dort.
Die \[C]einen raunten: ''Er ist ein Tier'',\[Am] die ander'n fragten: ''Was will der hier?''\[D] und dass er sich doch zum \[G]Teufel scher  ''So \[C]jagt ihn doch fort,\[E7] so \[Am]jagt ihn doch fort!''
\endverse

\beginverse\memorize
Sein \[Am]Haar in Strähnen und \[D]wirre, sein \[Am]Gang war gebeugt.
''Seht, dieser arme \[D]Irre ward vom \[Am]Teufel gezeugt.''
Der \[C]Pfarrer reichte ihm einen Krug voll \[Am]Milch, er trank in einem Zug.\[D] ''Der trinkt nicht vom Ge\[G]schirre, den hat die \[C]Wölfin gesäugt,\[E7]  den hat die \[Am]Wölfin gesäugt.''
\endverse

\beginverse
Mein ^Vater, der in ^unserem Orte ^Schulmeister war,
trat zu ihm hin, trotz ^böser Worte ^rings aus der Schar.
Er ^sprach zu ihm ganz ruhig und der ^Stumme öffnete den Mund und ^stammelte die ^Worte ''Heiße ^Kaspar,^ heiße ^Kaspar.''
\endverse 

\beginverse
Mein ^Vater brachte ihn ^mit nach Haus, ''Heiße ^Kaspar.''
Meine Mutter wusch seine ^Kleider aus und ^schnitt ihm das Haar.
^ Sprechen lehrte mein Vater ihn,^ lesen und schreiben und es schien, was man ihn ^lehrte, sog er in sich ^auf, wie ^gierig er war,^ wie ^gierig er war.
\endverse

\beginverse
Zur ^Schule gehörte der^zeit noch das ^Üttinger Feld.
Kaspar und ich, wir ^pflügten zu zweit, bald war ^alles bestellt.
Wir ^hegten und pflegten jeden Keim,^ brachten im Herbst die Ernte ein, von den ^Leuten vermale^deit, von ihren ^Hunden verbellt,^ von ihren ^Hunden verbellt.
\endverse

\beginverse
Ein ^Wintertag, der ^Schnee lag frisch, es war ^Januar.
Meine Mutter rief uns: ''^Kommt zu Tisch, das ^Essen ist gar.''
Mein ^Vater sagte: ''Appetit'',^ ich wartete auf Kaspars Schritt, mein ^Vater fragte ^mürrisch: ''Wo bleibt ^Kaspar,^ wo bleibt ^Kaspar?''
\endverse

\beginverse
Wir ^suchten und wir ^fanden ihn auf dem ^Pfad bei dem Feld.
Der Neuschnee wehte ^über ihn, sein Ge^sicht war entstellt.
Die ^Augen angstvoll aufgerissen,^ sein Hemd war blutig und zerrissen.^ Erstochen hatten sie ^ihn dort am ^Üttinger Feld,^ dort am ^Üttinger Feld.
\endverse

\beginverse
Der ^Polizeirat aus der ^Stadt füllte ^ein Formular.
''Gott nehm' ihn hin in ^seiner Gnad'', sagte ^der Herr Vikar.
Das ^Üttinger Feld liegt lang schon brach,^ nur manchmal bell'n mir noch die Hunde nach,^ dann streu' ich ein paar Blumen auf ^den Pfad für ^Kaspar,^ für ^Kaspar.
\endverse

\endsong

\beginscripture{}
Kaspar Hauser tauchte 1828 plötzlich in Nürnberg auf. Bei sich trug er einen Brief, laut diesem war er sein Leben lang eingesperrt.
1833 wurde er mit einer Stichwunde aufgefunden und starb drei Tage später. Ob er ermordet wurde oder sich selbst verletzte, sowie seine Herkunft, sind bis heute unklar.
\endscripture
