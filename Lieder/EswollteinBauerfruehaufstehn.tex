\beginsong{Es wollt' ein Bauer früh aufsteh'n}[
    wuw={Volkslied aus dem 15. Jahrhundert},
    mel={Zupfgeigenhansel},
    meljahr={1976},
    bo={142}, 
    pfii={76}, 
    pfiii={57}, 
    gruen={147},
]

\beginverse
\endverse
\includegraphics[draft=false, width=1\textwidth]{Noten/Lied042.pdf}

\beginverse
Und \[D]als der Bauer nach Hause kam, und als der Bauer nach \[A]Hause kam,
da wollt er was zu \[D]fressen ham'. \[A]Fal-te-rie-te-\[D]ra-la-la, \[A]fal-te-rie-te-\[D]ra.
\endverse

\beginverse
Ach ^Lieschen, koch mir Hirsebrei, ach Lieschen, koch mir ^Hirsebrei
mit Bratkartoffeln, ^Spiegelei! ^Fal-te-rie-te-^ra-la-la, ^fal-te-rie-te-^ra.
\endverse

\beginverse
Und ^als der Bauer saß und fraß, und als der Bauer ^saß und fraß,
da rumpelt in der ^Kammer was.  ^Fal-te-rie-te-^ra-la-la, ^fal-te-rie-te-^ra.
\endverse

\beginverse
Ach ^liebe Frau, was ist denn das? Ach liebe Frau, was ^ist denn das?
Da rumpelt in der ^Kammer was. ^Fal-te-rie-te-^ra-la-la, ^fal-te-rie-te-^ra.
\endverse

\beginverse
Ach ^lieber Mann, das ist der Wind, ach lieber Mann, das ^ist der Wind,
der raschelt da am ^Küchenspind. ^Fal-te-rie-te-^ra-la-la, ^fal-te-rie-te-^ra.
\endverse

\beginverse
Der ^Bauer sprach: Will selber seh'n. Der Bauer sprach: Will ^selber seh'n,
will selber in die ^Kammer geh'n.  ^Fal-te-rie-te-^ra-la-la, ^fal-te-rie-te-^ra.
\endverse

\beginverse
Und ^als der Bauer in d' Kammer kam, und als der Bauer in d' ^Kammer kam,
zog der Pfaff' die ^Hosen an.  ^Fal-te-rie-te-^ra-la-la, ^fal-te-rie-te-^ra.
\endverse

\beginverse
Ei ^Pfaff', was machst in meinem Haus? Ei Pfaff', was machst in ^meinem Haus?
Ich werf dich ja so^gleich hinaus. ^Fal-te-rie-te-^ra-la-la, ^fal-te-rie-te-^ra.
\endverse

\beginverse
Der ^Pfaff', der sprach: Was ich verricht', der Pfaff, der sprach: Was ^ich verricht',
dein' Frau, die kann die ^Beichte nicht.  ^Fal-te-rie-te-^ra-la-la, ^fal-te-rie-te-^ra.
\endverse

\beginverse
Da ^nahm der Bauer 'nen Ofenscheit, da nahm der Bauer 'nen ^Ofenscheit
und schlug den Pfaffen, ^dass er schreit. ^Fal-te-rie-te-^ra-la-la, ^fal-te-rie-te-^ra.
\endverse

\beginverse
Der ^Pfaffe schrie: O Schreck, o Graus!, der Pfaffe schrie: O ^Schreck, o Graus!
und hielt den Arsch zum ^Fenster raus. ^Fal-te-rie-te-^ra-la-la, ^fal-te-rie-te-^ra.
\endverse

\beginverse
Da ^kamen die Leut' von nah und fern, da kamen die Leut' von ^nah und fern
und dachten, es sei der ^Morgenstern. ^Fal-te-rie-te-^ra-la-la, ^fal-te-rie-te-^ra.
\endverse

\beginverse
Der ^Morgenstern, der war es nicht, der Morgenstern, der ^war es nicht,
es war des Pfaffen ^Arschgesicht. ^Fal-te-rie-te-^ra-la-la, ^fal-te-rie-te-^ra.
\endverse

\beginverse
So ^soll es allen Pfaffen geh'n, so soll es allen ^Pfaffen geh'n,
die nachts zu fremden ^Weibern geh'n. ^Fal-te-rie-te-^ra-la-la, ^fal-te-rie-te-^ra.
\endverse

\beginverse
Und ^die Moral von der Geschicht', und die Moral von ^der Geschicht':
Trau' nie des Pfaffen ^Arschgesicht! ^Fal-te-rie-te-^ra-la-la, ^fal-te-rie-te-^ra.
\endverse

\endsong

\beginscripture{}
Pfaffe ist ein wenig gebrauchtes Wort für einen Pfarrer beziehungsweise einen Geistlichen.
\endscripture
