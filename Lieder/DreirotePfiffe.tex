\beginsong{Drei rote Pfiffe}[
    txt={Heinz Rudolf Unger}, 
    mel={Georg Herrnstadt, Wilhelm Resetarits},
    jahr={1979}, 
    bo={198},
    siru={124},
    tonspur={340},
    index={Im Kreis ihrer Enkel, die alte Frau},
]

\beginverse
\endverse
\includegraphics[draft=false, width=1\textwidth]{Noten/Lied030.pdf}	

\beginverse
Die \[D]Drau hinunter trieb Mond um Mond, es \[Hm]brach der Faschistenkrieg aus.
Da \[F\#m]hatte ich einen \[G]Mann an der Front und \[Hm]hatte drei Kinder im \[A]Haus,
und hatte drei Kinder im \[D]Haus.
Wie \[Hm]tönte da markiger Nazigesang von \[A]deutschem Boden und Blut.
\[G]Manch ein Bursch' in die \[D]Berge entsprang, ich trug \[F\#m]Flugblätter unter dem Hut,
ich trug \[G]Flugblätter unter dem \[D]Hut.
Der \[D]Gestapo war kalt und der Gauleiter schalt: Parti\[Hm]sanen im eigenen Land!
Ich \[F\#m]trug das Geflüster und \[G]Brot in den Wald. Sie \[Hm]haben mich Jelka ge\[A]nannt.
Sie haben mich Jelka ge\[D]nannt.
\endverse

\beginchorus
Ver\[Dm]schwiegene Bäume, ver\[B]schworener Wald
Und \[F]drei rote Pfiffe, \[Am]drei rote Pfiffe, \[B]drei rote \[Gm]Pfiffe im \[D]Wald.
\endchorus

\beginverse
Der ^Winter war nass und uns wärmte der Hass, viele ^sind's, die die Erde heut' birgt. Wir ^haben gefochten, dort ^oben am Pass, an ^uns'rer Befreiung ge^wirkt,
an uns'rer Befreiung ge^wirkt.
Der ^Krieg war vorbei, da war Stille im Land, da ^waren die Lautesten leis'.
Sie ^nahmen das Hitler^bild von der Wand, ihre ^Westen, die wuschen sie weiß,
ihre ^Westen, die wuschen sie ^weiß.
^Ihr, meine Enkel, was hört ihr so stumm die ^alten, die kalten Berichte?
Jetzt ^trampeln sie wieder auf euren ^Rechten herum, er^innert euch meiner Ge^schichte, erinnert euch meiner Ge^schichte.
\endverse

\printchorus

\endsong

\beginscripture{}
% Siehe Seite 160
\endscripture
