\beginsong{Die freie Republik}[
    wuw={Studentengruppen},
    jahr={1937}, 
    bo={206}, 
    pfii={47}, 
    pfiii={23}, 
    gruen={183}, 
    kssiv={22}, 
    siru={132}, 
    tonspur={348}, 
    index={In dem Kerker saßen},
]

\beginverse
\endverse
\includegraphics[draft=false, width=1\textwidth]{Noten/Lied022.pdf}	

\beginverse
\[C]Und der Kerkermeister \[G7]sprach es täglich \[C]aus:
''Sie, Herr Bürgermeister, es \[G7]reißt mir keiner \[C]aus!''
\lrep Aber \[F]doch sind sie verschwunden abends aus dem \[C]Turm,
um die zwölfte Stunde, \[G7]bei dem großen \[C]Sturm.\rrep
\endverse

\beginverse
^Und am ander'n Morgen ^hört man den A^larm.
Oh, es war entsetzlich, ^der Soldaten^schwarm!
\lrep Sie ^suchten auf und nieder, sie suchten hin und ^her,
sie suchten sechs Studenten und ^fanden sie nicht ^mehr.\rrep
\endverse

\beginverse
^Doch sie kamen wieder, mit ^Schwertern in der ^Hand.
''Auf, auf, ihr deutschen Brüder, jetzt ^geht's für's Vater^land!
\lrep Jetzt ^geht's für Menschenrechte und für das Bürger^glück!
Wir sind doch keine Knechte der ^freien Repu^blik.''\rrep
\endverse

\beginverse
Wenn ^euch die Leute fragen: ''^Wo ist Absa^lom?''
So dürfet ihr wohl sagen: ''^Oh, der hänget ^schon.''
\lrep Er ^hängt an keinem Baume, er hängt an keinem ^Strick,
sondern an dem Glauben der ^freien Repu^blik.\rrep
\endverse


\endsong

\beginscripture{}
Erkennungslied der demokratischen Bewegung im Vormärz. Text und Melodie entstanden nach der Flucht von sechs Studenten, die am gescheiterten Frankfurter Wachensturm vom 3. April 1833 beteiligt waren. Ziel des Aufstands war die Ausrufung einer deutschen Republik durch die Besetzung der Hauptwache und Konstablerwache in Frankfurt am Main. Die Flucht der Inhaftierten gelang mit Hilfe sympathisierender Gefängniswärter. Infolge des Umsturzversuchs wurden über 1800 Personen gesucht; 39 wurden zum Tode oder zu lebenslanger Haft verurteilt.
\endscripture
