\beginsong{Ballade vom roten Haar}[
    index={Im Sommer war das Gras so tief},
    txt={Paul Zech},
    txtjahr={1931},
    mel={Peter Rohland},
    meljahr={1960er},
]

\beginverse  
\endverse
\includegraphics[draft=false, width=1\textwidth]{Noten/BalladeVomRotenHaar.pdf}

\beginverse\memorize
Im \[Am]Feld den ganzen \[Dm]Sommer war der rote \[G]Mohn so rot nicht wie dein \[C]Haar,
jetzt wird es \[Am]abge\[Em]mäht - das \[G]Gras, die bunten \[F]Blumen welken auch da\[Am]hin.
Und \[Am]wenn der rote \[Dm]Mohn so blass geworden \[G]ist dann hat es keinen \[C]Sinn,
\lrep dass es noch \[Am]weiße \[Em]Wolken \[G]gibt, ich hab' mich \[F]in dein rotes Haar ver\[Am]liebt.\rrep
\endverse

\beginverse
Du ^sagst, dass es bald ^Kinder gibt, wenn man sich ^in dein rotes Haar ver^liebt,
so rot wie ^Mohn, so ^weiß wie ^Schnee, im Herbst, da ^kehren viele Wunder ^ein.
\mbox{Wa^rum soll's auch bei ^uns nicht sein, du bliebst im ^Winter doch mein rotes ^Reh,} %mbox needed to keep Text in one line.
\mbox{\lrep und wenn es ^tausend ^Schön're ^gibt, ich hab' mich ^in dein rotes Haar ver^liebt.\rrep} %mbox needed to keep Text in one line.
\endverse

\endsong

\beginscripture{}
Der Originaltext stammt von François Villon, einem französischen Dichter des Spätmittelalters. Er verfasste das Gedicht im 15. Jahrhundert.
\endscripture