\beginsong{Der Mond ist aufgegangen}[
    txt={Matthias Claudius}, 
    txtjahr={1778},
    mel={Johann Abraham Peter Schulz}, 
    meljahr={1790},
    pfiii={34}, 
    siru={44}, 
    tonspur={22},
]

\beginverse
\endverse 
\includegraphics[draft=false, width=1\textwidth]{Noten/Lied017.pdf}	

\beginverse
\[D]Wie \[A7]ist \[D]die \[G]Welt \[D]so \[A7]stil\[D]le und in der \[G]Dämm'\[D]rung \[A7]Höl\[D]le
so traulich \[G]und so \[A]hold! 
\[D]Als \[A7]ei\[D]ne \[G]stil\[D]le \[A7]Kam\[D]mer, wo ihr des \[G]Ta\[D]ges \[A7]Jam\[D]mer
verschlafen \[G]und \[D]ver\[A7]gessen \[D]sollt.
\endverse

\beginverse
^Seht ^ihr ^den ^Mond ^dort ^ste^hen? Er ist nur ^halb ^zu ^se^hen
und ist doch ^rund und ^schön.
^So ^sind ^wohl ^man^che ^Sa^chen, die wir ge^trost ^be^lach^en,
weil ^uns're ^Augen ^sie nicht ^seh'n.
\endverse

\beginverse
^Wir ^stol^zen ^Men^schen^kin^der sind eitel ^ar^me ^Sün^der
und wissen ^gar nicht ^viel.
^Wir ^spin^nen ^Luft^ge^spin^ste und suchen ^vie^le ^Kün^ste
und kommen ^wei^ter ^von dem ^Ziel.
\endverse

\beginverse
^Gott, ^lass ^dein ^Heil ^uns ^schau^en, auf nichts Ver^gang^lich's ^trau^en,
nicht Eitel^keit uns ^freu'n!
^Lass ^uns ^ein^fal^tig ^wer^den, und vor dir ^hier ^auf ^Er^den
wie Kinder ^fromm ^und ^fröhlich ^sein!
\endverse

\beginverse
^Wollst ^end^lich ^son^der ^Grä^men aus dieser ^Welt ^uns ^neh^men
durch einen ^sanften ^Tod.
^Und ^wenn ^du ^uns ^ge^nom^men, lass uns in ^Him^mel ^kom^men,
du unser ^Herr ^und ^unser ^Gott!
\endverse

\beginverse
^So ^legt ^euch ^denn, ^ihr ^Brü^der, in Gottes ^Na^men ^nie^der,
kalt ist der ^Abend^hauch.
^Ver^schon' ^uns, ^Gott, ^mit ^Stra^fen und lass uns ^ru^hig ^schla^fen
und unser'n ^kran^ken ^Nachbar ^auch!
\endverse

\endsong

\beginscripture{}
Dem Gedicht von Claudius als Vorlage diente "Nun ruhen alle Wälder" (1653) von Paul Gerhardt. Caulius' Todesgedicht vor dem Hintergrund der Heilserwartung eines Christen könnte auch schon vor 1778 in Darmstadt entstanden sein. Seine Einfachheit wurde teilweise als naiv kritisiert, erfreute sich jedoch schnell großer Beliebtheit in der deutschen Bevölkerung.
\endscripture
