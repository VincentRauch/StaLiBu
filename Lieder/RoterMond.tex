\beginsong{Roter Mond}[
    wuw={Anja Klenk (Hortenring Ernsthofen)}, 
    jahr={1980}, 
    bo={266}, 
    pfii={90}, 
    pfiii={27}, 
    gruen={68}, 
    kssiv={38}, 
    siru={194},
    tonspur={436}, 
]

\transpose{-7}

\beginverse
\endverse
\includegraphics[draft=false, width=1\textwidth]{Noten/RoterMond.pdf}

\beginverse
\[Em]Sterne ste\[D]h'n hell am Firmament, \[Em]solche Nac\[D]ht findet nie ein End'.
\lrep \[G]Dieses La\[D]nd, wild und schön, und \[Am]wir dürfen seine \[Em]Herrlichkeit seh'n. \rrep
\endverse

\beginverse
^Rauher Fe^ls, Moos und Heidekraut, ^weit entfer^nt schon der Morgen graut.
\lrep ^Fahne we^ht, weiß und blau, das ^Gras schimmert unterm ^Morgentau. \rrep
\endverse

\beginverse
\[Em]Fahrt vorb\[D]ei, morgen geht es fort, \[Em]kommen w\[D]ir wieder an den Ort?
\[G]Norden i\[D]st unser Glück und \[Am]in uns bleibt nur die Er\[Em]innerung zurück.
\[G]Norden i\[D]st unser Glück und \[Am]wir schwören auf ein \[Em]neues zurück.
\endverse

\endsong

\beginscripture{}
Dieses Lied ist 1980 bei einem Pfadfinderlager des Hortenring Ernsthofen in Schweden entstanden. Die Farben~ ''weiß und blau'' beziehen sich auf das Banner des Pfadfinderbundes. Dieser Bund wurde Ende der Neunziger Jahre aufgelöst.
Wenn der Mond nah am Horizont steht, dann wird das Licht so durch die Atmosphäre gefiltert, dass er rot erscheint.
\endscripture 
