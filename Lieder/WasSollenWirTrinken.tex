\beginsong{Was sollen wir trinken?}[
]

\beginverse
\endverse
\includegraphics[draft=false, width=1\textwidth]{Noten/WasSollenWirTrinken.pdf}

\beginverse
\lrep \[D] Dann wollen wir \[Em]schaffen, sieben Tage \[D]lang,
dann wollen wir \[Em]schaffen, \[D]Hand in \[Em]Hand \rrep
\lrep \[D] Es gibt ge\[G]nug für \[D]uns zu \[G]tun,
\[D] d'rum lasst uns \[Em]schaffen, jeder packt mit \[D]an,
wir schaffen zu\[Em]sammen, \[D]nicht al\[Em]lein. \rrep
\endverse

\beginverse
\lrep ^ Erst müssen wir ^kämpfen, keiner weiß wie ^lang,
Erst müssen wir ^kämpfen für ^unser ^Ziel. \rrep
\lrep ^ Und für das ^Glück von ^jeder^mann,
^ d'rum lasst uns ^kämpfen, los, fangt heute ^an,
wir kämpfen zu^sammen, ^nicht al^lein. \rrep
\endverse

\beginverse
\lrep ^ Dann wollen wir ^trinken, sieben Tage ^lang,
dann wollen wir ^trinken, wir ^haben ^Durst. \rrep
\lrep ^ Es ist ge^nug für ^alle ^da,
^ d'rum lasst uns ^trinken, rollt das Fass her^rein,
wir trinken zu^sammen, ^nicht al^lein. \rrep
\endverse

\endsong

\beginscripture{}
Original stammend aus dem bretonischen Volkslied 'Son ar christr' aus 1929. Seit den 1970ern wurde das Lied mit unterschiedlichen Texten sehr oft gecovert, zum Beispiel von den 'Bots' als Arbeiter- und Friedenslied. Weitere Coverversionen existerien u.a. von Oktoberklub (DDR, 1977), Scooter (1998), Mickie Krause (2008) und K.I.Z (2008).
\endscripture