\beginsong{Gregor}[
    txt={aus der Jungenschaft, niedergeschrieben von Eberhard Köbel und Günther Wolff}, 
    txtjahr={1934}, 
    mel={nach ukrainischem Volkslied}, 
    bo={166},  
    pfii={104}, 
    pfiii={65}, 
    kssiv={40}, 
    siru={87}, 
    tonspur={278}, 
    index={Gehe nicht, oh Gregor},
]

\beginverse
\endverse
\includegraphics[draft=false, width=1\textwidth]{Noten/Lied048.pdf}	

\beginverse
\[Dm]Dort ist auch die eine mit den \[A7]schwarzen Augen\[Dm]brauen.
Glaube mir, oh Gregor, das ist \[A7]eine Zaube\[Dm]rin. \[C]
\lrep \[F]Ihre schmale Hand braut dir \[C]Tee aus Zauber\[F]kräu\[A7]tern,
\[Dm]legt sich über deine Seele \[A7]wie der Herbst auf's \[Dm]Land. \[(C)]\rrep
\endverse

\beginverse
^Sonntag früh beim Glockenläuten ^grub sie aus das ^Kraut,
schnitt es Montag, alle Sünden ^hexte sie hi^nein, ^
\lrep ^holt' es Dienstag vor, braute ^Zaubertrank aus den ^Kräu^tern,
^Mittwoch Nacht beim Reigentanze ^gab sie ihn Gre^gor. \[(C)]\rrep
\endverse

\beginverse
^Und am Tag darauf am Tage ^war Grischenko ^tot.
Freitag kam voll Leid und Klage ^und beim Abend^rot ^
\lrep ^trug man ihn zur Ruh' an der ^Grenze an der ^Stra^ße.
^Viele fromme Leute kamen, ^sahen traurig ^zu. \[(C)]\rrep
\endverse

\beginverse
^Viele Knaben, viele Burschen ^klagten um Gre^gor.
Böse Hexe, Zauberhexe, ^schwarze Zauber^frau, ^
\lrep ^deine Augenbrauen werden ^keinen mehr be^tö^ren,
^nie mehr wird ein zweiter Gregor ^deinen Künsten ^trau'n. \[(C)]\rrep
\endverse


\endsong

\beginscripture{}
Es handelt sich in dem Lied wahrscheinlich lediglich um ein Mädchen, das den geliebten Gregor vergiftet, weil es auf andere Frauen eifersüchtig ist.
\endscripture
