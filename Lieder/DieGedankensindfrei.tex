\beginsong{Die Gedanken sind frei}[
    txt={Flugblätter aus Bern},
    txtjahr={ca. 1780},
    mel={aus 'Lieder der Brienzer Mädchen', ca. 1810},
    pfii={115}, 
    pfiii={31}, 
    kssiv={71}, 
    siru={52},
    tonspur={206}, 
    buedel={67}, 
]

\beginverse
\endverse
\includegraphics[draft=false, width=1\textwidth]{Noten/Lied023.pdf}

\beginverse
Ich \[G]denke, was ich will und \[D7]was mich be\[G]glücket, 
und alles in der Still', und \[D7]wie es sich \[G]schicket.
Mein \[D7]Wunsch und Be\[G]gehren kann \[D7]niemand ver\[G]wehren.
Es \[C]bleibet da\[G]bei: Die Ge\[D7]danken sind \[G]frei!
\endverse 

\beginverse
Und ^sperrt man mich ein im ^finsteren ^Kerker,
das alles sind rein ver^gebliche ^Werke,
denn ^meine Ge^danken zer^reißen die ^Schranken
und ^Mauern ent^zwei: Die Ge^danken sind ^frei!
\endverse

\beginverse
Nun ^will ich auf immer den ^Sorgen ent^sagen
und will mich auch nimmer mit ^Grillen mehr ^plagen.
Man ^kann ja im ^Herzen stets ^lachen und ^scherzen
und ^denken da^bei: Die Ge^danken sind ^frei!
\endverse

\beginverse
Ich ^liebe den Wein, mein ^Mädchen vor ^allen,
die tut mir allein am ^Besten ge^fallen.
Ich ^bin nicht al^leine bei ^einem Glas ^Weine;
mein ^Mädchen da^bei: Die Ge^danken sind ^frei!
\endverse

\endsong

\beginscripture{}
Der Text des Liedes stamm von anonymen Flugblättern aus Bern aus dem Jahr 1780. Mit Melodie wurde das Lied zuerst in der Sammlung~ ''Lieder der Brienzer Mädchen'' im frühend 19. Jahrhundert in Bern gedruckt. Zur Verbreitung trugen maßgeblich die Im Jahr 1842 veröffentlichten~ ''Schlesischen Volkslieder'' von Hoffmann von Fallersleben und Ernst Richter bei.

Das Lied diente in Zeiten politischer Unterdrückung immer wieder als Ausdruck der Sehnsucht nach Freiheit und Unabhängigkeit, so etwa während der Studentenbewegung des Vormärz, im Widerstand gegen den Nationalsozialismus oder während der Berlin-Blockade von 1948.
\endscripture
