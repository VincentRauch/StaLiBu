\beginsong{Edelweißpiraten}[
    %wuw={Herwig Steymans, Hans-Jörg Maucksch},
    %bo={280},
    %pfii={70},
    %pfiii={26},
    %siru={209},
    %tonspur={466},
    index={Sie saßen oft am Märchensee},
]

\beginverse
\[G] Sie saßen \[Hm]oft am Märchen\[C]see beim Lager\[G]feuer,
\[Em] sie wollten \[Am]leben, wie es \[D]ihnen ge\[G]fiel.
Der neue \[Hm]Kurs im deutschen \[C]Land war nicht ge\[G]heuer,
\[Em] sie wollten \[Am]frei sein mit Ge\[D]sang, Gitarren\[G]spiel.
\[D] Mit ihrer \[C]Kleidung nahmen \[G]sie's nicht so ge\[D]nau.
Ganz offen \[C]trugen sie das \[G]Edelweiß zur \[D]Schau,
und das war \[C]gut, sie hatten \[G]Mut.
\endverse

\beginverse
\[G] Sie hatten \[Hm]nichts im Sinn mit \[C]braunen Nazi\[G]horden,
\[Em] sie hielten \[Am]nichts von dem Ge\[D]schrei nach Heil und \[G]Sieg.
Was war denn \[Hm]bloß aus ihrem \[C]Vaterland ge\[G]worden?
\[Em] Man schürte \[Am]offen den ver\[D]brecherischen \[G]Krieg.
\[D] Da gab's nur \[C]eins zu tun: Be\[G]frei'n wir dieses \[D]Land!
Da durfte \[C]keiner ruh'n, wir \[G]leisten Wider\[D]stand!
Sie hatten \[C]Mut und das war \[G]gut.
\endverse

\beginchorus
\lrep Vielleicht wird \[D]morgen schon\[Am] eine neue \[C]Zeit anfangen,
\[G] vielleicht ist \[D]morgen schon\[C] der Spuk vor\[G]bei. \rrep
\endchorus

\beginverse
^ Da gab's 'nen ^Güterzug mit ^Kriegsgerät und ^Waffen
^ und was man ^sonst noch braucht für ^einen Völker^mord.
Da machten ^sie sich an den ^Gleisen kurz zu ^schaffen,
^ der Zug er^reichte niemals ^den Bestimmungs^ort. \newpage
^ Und Essens^marken vom Par^teibüro der ^Stadt
war'n plötzlich ^weg und Zwangsar^beiter wurden ^satt,
und das war ^gut, und das war ^gut.
\endverse

\beginverse
^ Sie glaubten ^fest daran, dass ^sie den Sieg er^ringen,
^ sie glaubten ^fest daran, aus ^Schaden wird man ^klug.
Sie glaubten ^fest daran, als ^sie zum Galgen ^gingen,
^ sie glaubten ^fest daran, als ^man sie vorher ^schlug.
^ Und diese ^Angst, die hinter ^jeder Folter ^steht,
die ist so ^groß, dass man den ^besten Freund ver^rät,
versteht man ^gut, versteht man ^gut.
\endverse

\beginchorus
\lrep Vielleicht wird \[D]morgen schon\[Am] eine neue \[C]Zeit anfangen,
\[G] vielleicht ist \[D]morgen schon\[C] der Spuk vor\[G]bei. \rrep
\endchorus

\beginverse
^ Sie stehen ^heute noch auf ^manchen schwarzen ^Listen,
^ ich würd' fast ^sagen, es ist's ^wieder mal so^weit.
In Amt und ^Würde sitzen ^immer noch Fa^schisten,
^ und zum to^talen Krieg ist ^mancher schnell be^reit.
^ Doch seh' ich ^Tausende, und ^das beruhigt mich ^sehr,
die zeigen ^offen das zer^brochene Ge^wehr,
und das macht ^Mut, und das macht ^Mut.
\endverse

\beginchorus
\lrep Und dann wird \[D]morgen schon\[Am] eine neue \[C]Zeit anfangen
\[G] und dann ist \[D]morgen schon\[C] der Spuk vor\[G]bei. \rrep
\endchorus

\endsong

%\beginscripture{}
%Die Edelweißpiraten waren eine Gruppe Jugendlicher zur Zeit des Dritten Reichs, die sich der Hitlerjugend nicht anschließen wollten. Als sie weiterhin auf Fahrten gingen, wurden sie von der Gestapo verfolgt und wurden oppositionell politisch. Bis zuletzt wurden Edelweißpiraten, die an einem Abzeichen erkennbar waren, verfolgt und erschossen. Bis heute sind sie nicht als Widerstandsgruppe anerkannt. Dieses Lied reflektiert besonders die Bedeutung der Ehrenfelder Gruppe.
%\endscripture
