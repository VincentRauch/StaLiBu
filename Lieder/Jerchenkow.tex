\beginsong{Jerchenkow}[
    txt={deutsch von Peer (Dieter Krolle)},
    txtjahr={1958}, 
    mel={nach einem ukrainischen Volkslied}, 
    bo={214}, 
    pfii={40}, 
    pfiii={18}, 
    siru={148}, 
    tonspur={362}, 
    index={Jeden Abend träumt Jerchenkow},
]

\beginverse
\endverse
\includegraphics[draft=false, width=1\textwidth]{Noten/Lied057.pdf}	

\beginverse*
{\nolyrics \[Dm] \[Am] \[E] \[Am] \[Dm] \[Am] \[E] \[Am-E-Am]}
\endverse

\beginverse\memorize
Als der \[Am]Mond stand nachts am Himmel, klopften \[E]wir beim Starosten \[Am]an.
Alles klauten wir dem Lümmel, selbst den \[E]roten Sara\[Am]fan.
\endverse

\beginchorus
Man \[Dm]müsste wieder \[Am]zwei Pistolen \[E]und ein Pferdchen \[Am]haben,
da\[Dm]zu mit einer \[Am]Reiterschar nach \[E]Nischnij Nowgorod \[Am]traben.
Man \[Dm]müsste wieder \[Am]zwei Pistolen \[E]und ein Pferdchen \[Am]haben,
da\[Dm]zu mit einer \[Am]Reiterschar nach \[E]Nischnij Nowgorod \[Am]tr\[E]ab\[Am]en.
\nolyrics \[Dm] \[Am] \[E] \[Am] \[Dm] \[Am] \[E] \[Am-E-Am]
\endchorus

\beginverse 
Dreimal ^ritt ich nach Odessa, dreimal ^sah ich Peters^burg
als des Zaren Leibkosaken unter ^Hebnio Sara^tow.
\endverse

\printchorus

\beginverse
An die ^vielen langen Feste denk' ich ^sehnsuchtsvoll zu^rück;
Vodkafässer, Tanzen, Singen, diese ^Zeit kehrt nie zu^rück.
\endverse

\printchorus

\endsong

\beginscripture{}
Nischnij (auch: Nischni) Nowgorod = fünftgrößte russische Stadt, Industrie-Metropole; Starost = Vermögen Verwaltender (etwa: Vorsteher)
\endscripture