\beginsong{Ballade vom roten Haar}[
    wuw={François Villon},
    jahr={1431-1463},
]

\beginverse
Im \[Am]Sommer war das \[Dm]Gras so tief, dass jeder \[G]Wind daran vorüber \[C]lief.
Ich habe \[Am]dort dein \[Em]Blut ge\[G]spürt und wie es \[F]heiß zu mir herüber \[Am]rann.
Du \[Am]hast nur mein Ge\[Dm]sicht berührt, da starb er \[G]einfach hin, der harte \[C]Mann.
Weil's solche \[Am]Liebe \[Em]nicht mehr \[G]gibt, ich hab' mich \[F]in dein rotes Haar ver\[Am]liebt.
\endverse

\beginverse
Im ^Feld, den ganzen ^Sommer war der rote ^Mohn so rot nicht wie dein ^Haar.
Jetzt wird es ^abge^mäht, das ^Gras, die bunten ^Blumen welken auch da^hin.
Und ^wenn der rote ^Mohn so blass geworden ^ist, dann hat es keinen ^Sinn,
dass es noch ^weiße ^Wolken ^gibt. Ich hab' mich ^in dein rotes Haar ver^liebt.
\endverse

\beginverse
Du ^sagst, daß es bald ^Kinder gibt, wenn man sich ^in dein rotes Haar ver^liebt.
So rot wie ^Mohn, so ^weiß wie ^Schnee, im Herbst, da ^kehren viele Wunder ^ein.
Wa^rum soll's auch bei ^uns nicht sein, du bliebst im ^Winter doch mein rotes ^Reh,
und wenn es ^tausend ^Schönre ^gibt, ich hab' mich ^in dein rotes Haar ver^liebt.
\endverse

\endsong