\beginsong{Ballade vom roten Haar}[
    wuw={François Villon},
    jahr={1431-1463},
    tonspur={570}, 
    index={Roter Mohn},
    index={Im Sommer war das Gras so tief},
]

\beginverse
\endverse

\beginverse
Im \[Am]Sommer war das \[Dm]Gras so tief, dass jeder \[G]Wind daran vorüber \[C]lief.
Sie \[G]banden mir die Händ' und mein \[Am]Haus, es hat gebrennt.
Ich \[Am]sah den Galgen steh'n. Sie \[C]zwangen mich zu geh'n.
Sie \[G]wollten meinen Tod, keiner \[Am]half mir in der Not.
\endverse

\beginchorus
\lrep \[Am]Wenn der Nebel auf das \[D]Moor sich senkt, der \[F]Piet am \[G]Galgen \[Am]hängt.\rrep
\endchorus

\beginverse
Was ^kratzt da am Genick? Ich ^spür' den rauhen Strick.
Ein ^Mönch der betet dort und spricht ^für mich fromme Wort,
die ^Wort, die ich nicht kenn', wer ^lehrte sie mich denn? 
Fünf ^Raben fliegen her, doch ich ^sehe sie nicht mehr.
\endverse

\beginchorus
\lrep \[Am]Wenn der Nebel auf das \[D]Moor sich senkt, der \[F]Piet am \[G]Galgen \[Am]hängt.\rrep
\endchorus

\endsong

\beginscripture{}
Der Verfasser lässt im Unklaren, auf wen sich die Person des Piet tatsächlich bezieht. Es sind jedoch Parallelen zum Schinderhannes erkennbar.
\endscripture
