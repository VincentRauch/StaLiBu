\beginsong{Gospodar}[
    mel={Rudi Rogoll, 1984}, 
    txt={Theodor Kramer, 1984}, 
    bo={172}, 
    pfiii={64}, 
    siru={90},
    tonspur={284}, 
]

\beginverse
\endverse
\includegraphics[draft=false, width=1\textwidth]{Noten/Lied047.pdf}	

\beginverse
\[Am]Treiben wir die \[Dm]Fremden \[Am]über's \[E]Jahr erst \[Am]aus.
\[C]Gospodar, wer glaubst du, \[G]bleibt im Herrschafts\[C]haus?
\lrep \[Am]Werd' ich knechtisch \[Dm]aufsteh'n, \[G7]wo ich mächtig \[C]sitz'?
\[Am]Sind nicht solche \[Dm]Tölpel, \[Am]Bruder \[E]Sliwo\[Am]witz. \rrep
\endverse

\beginverse
^Haben unser ^Herzblut ^nicht für ^nichts ver^tan.
^Alles für die Seinen ^will der Parti^san:
\lrep ^Mutterschaf und ^Lämmer, ^Gänse, Geiß und ^Kitz,
^Kürbis und Me^lonen, ^Mais und ^Sliwo^witz. \rrep
\endverse

\beginverse
^Sind die wilden ^Schweine ^aus dem ^Land ver^jagt,
^die verkohlten Hütten ^aufgebaut und ^ragt
\lrep ^blank im Dorf der ^Maibaum, ^Flattern und Ge^fitz;
^oh, wie wird das ^schön sein, ^Bruder ^Sliwo^witz. \rrep
\endverse

\endsong

\beginscripture{}
Partisanen = Bewaffnete Widerstandskämpfer, die im eigenen Land gegen eine Besatzungsmacht, ein autoritäres Regime oder fremde Truppen kämpfen.
Besonders während der Besetzung durch die Achsenmächte im Zweiten Weltkrieg spielten Partisanen eine bedeutende Rolle im Widerstand.

Gospodar kommt aus dem slawischen und bedeutet Gaststätte. Sliwowitz ist ein in Osteuropa verbreiteter Pflaumenbranntwein.
\endscripture
