 \beginsong{Bella Ciao}[
    txt={Horst Berner},
    pfii={101}, 
    pfiii={30}, 
    tonspur={246}, 
    siru={65}, 
    index={Eines Morgens, in aller Frühe},
]

 \beginverse
 \endverse
 \includegraphics[draft=false, width=1\textwidth]{Noten/BellaCiaoV.pdf}

\beginverse
Parti\[Em]sanen, kommt, nehmt mich mit euch,
o bella ciao, bella ciao, bella \[H7]ciao, ciao, ciao,
\lrep Parti\[Am]sanen, kommt, nehmt mich \[Em]mit euch,
denn ich \[H7]fühl', der Tod ist \[Em]nah. \rrep
\endverse

\beginverse
Wenn ich ^sterbe, oh ihr Genossen,
o bella ciao, bella ciao, bella ^ciao, ciao, ciao,
\lrep wenn ich ^sterbe, oh ihr Ge^nossen,
Bringt mich ^dann zur letzten ^Ruh'. \rrep
\endverse

\beginverse
In den ^Schatten der kleinen Blume,
o bella ciao, bella ciao, bella ^ciao, ciao, ciao,
\lrep in den ^Schatten der kleinen ^Blume,
in die ^Berge bringt mich ^dann. \rrep
\endverse

\beginverse
Und die ^Leute, die geh'n vorüber,
o bella ciao, bella ciao, bella ^ciao, ciao, ciao,
\lrep und die ^Leute, die geh'n vo^rüber,
seh'n die ^kleine Blume ^steh'n. \rrep
\endverse

\beginverse
Diese ^Blume, so sagen alle,
o bella ciao, bella ciao, bella ^ciao, ciao, ciao,
\lrep ist die ^Blume des Parti^sanen,
der für ^uns're Freiheit ^starb. \rrep
\endverse

\endsong

\beginscripture{}
Die deutsche Adaption eines italienischen Partisanenliedes, das ursprünglich bereits zu Anfang des 20. Jahrhunderts von Reispflückerinnen in der Po-Ebene gesungen und während des Zweiten Weltkriegs zur Hymne des Widerstands gegen Faschismus wurde.
Die deutsche Version von Horst Berner wurde in der DDR verbreitet und diente als Ausdruck antifaschistischer Solidarität.
\endscripture
