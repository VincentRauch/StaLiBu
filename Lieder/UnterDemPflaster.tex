\beginsong{Unter dem Pflaster}[
    index={Komm lass dich nicht erweichen},
    wuw={Schneewittchen},
    jahr={1978},
]

\beginverse
\endverse
\includegraphics[draft=false, width=1\textwidth]{Noten/UnterDemPflaster.pdf}

\beginverse
\[A] Komm lass dir \[F\#m]nicht erzählen, \[Hm] was du zu \[E]lassen hast.
\[A] Du kannst doch \[F\#m]selber wählen, \[Hm] nur langsam, \[E]keine Hast
\endverse

\beginchorus
\[Am] Unter dem \[G]Pflaster, ja \[Am]da \[G]liegt der \[Am]Strand,
\[Am] komm reiß auch \[G]du ein paar \[Am]Steine \[G]aus dem \[Am]Sand.
\endchorus

\beginverse
\[A] Zieh die Schuhe \[F\#m]aus, \[Hm] die schon so \[E]lang dich drücken.
\[A] Lieber barfuß \[F\#m]lauf, \[Hm] aber nicht auf \[E]ihren Krücken.
\endverse

\beginchorus
\[Am] Unter dem \[G]Pflaster, ja \[Am]da \[G]liegt der \[Am]Strand,
\[Am] komm reiß auch \[G]du ein paar \[Am]Steine \[G]aus dem \[Am]Sand.
\endchorus

\newpage

\beginverse
\[A] Dreh dich und \[F\#m]tanz, \[Hm] damit sie \[E]dich nicht packen.
\[A] Verscheuch' sie \[F\#m]ganz \[Hm] mit deinem \[E]lauten Lachen.
\endverse

\beginchorus
\[Am] Unter dem \[G]Pflaster, ja \[Am]da \[G]liegt der \[Am]Strand,
\[Am] komm reiß auch \[G]du ein paar \[Am]Steine \[G]aus dem \[Am]Sand.
\endchorus

\beginverse
\[A] Die größte \[F\#m]Kraft \[Hm] ist deine \[E]Fantasie.
\[A] Wird die Ketten \[F\#m]weg \[Hm] und schmeiß sie \[E]gegen die,
\[A] die mit ihrer \[F\#m]Macht \[Hm] deine Kräfte \[E]brechen wollen.
\endverse

\beginchorus
\[Am] Unter dem \[G]Pflaster, ja \[Am]da \[G]liegt der \[Am]Strand,
\[Am] komm reiß auch \[G]du ein paar \[Am]Steine \[G]aus dem \[Am]Sand.
\endchorus

\endsong

\beginscripture{}
''Sous les pavés, la plage!'' (unter den Pflastersteinen, der Strand) war ein beliebter Spruch der Mai-Revolte in Paris und wurde von Arbeitern und Studenten, welche die festgefahrenen Strukturen anprangerten, an die Mauern geschrieben.~ ''Es ist ein [...] Emanzipationslied, nicht nur für Frauen. Die Steine sollen nicht zum werfen benutzt werden, sondern der Sand unter den Steinen zum Tanzen frei gelegt werden.'' (Angi Doomdey, Schneewittchen)
\endscripture